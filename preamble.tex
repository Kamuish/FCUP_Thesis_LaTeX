%-------------------------------------------------------------------------
%	PACKAGES AND OTHER DOCUMENT CONFIGURATIONS
%-------------------------------------------------------------------------

% Include pdf pages in the document
% Necessary to include the front pages (cover and etc.)
\usepackage{pdfpages}

% Fix top page geometry on long titles
\setlength{\headheight}{14pt}  %Try fix error

% Language hyphenation and typographical rules
\usepackage[portuguese,english]{babel}
%Custom hyphenization
\hyphenation{Py-thon}
\hyphenation{Ju-py-ter}
\hyphenation{Ma-the-ma-ti-ca}

% Inline quotes
% added for \begin{displayquote}
\usepackage[autostyle]{csquotes}

% Bibliography setup
% Use the natbib reference package - read up on this to edit the reference
% style; if you want text (e.g. Smith et al., 2012) for the in-text references
% (instead of numbers), remove 'numbers'
\usepackage[square, numbers, comma, sort&compress]{natbib}
\bibliographystyle{IEEEtranN}  % I actually quite like this one
% \bibliographystyle{apsrev4-1-etal} % With emphasized titles. ORIGINAL
% Prevent that the first citation is in the ToC
\usepackage{notoccite}

% Interesting float placements (like 'H') and custom float types
\usepackage{float}
% Text wrapped around pictures
% https://pt.sharelatex.com/learn/Wrapping_text_around_figures
\usepackage{wrapfig}
% Force float barriers, use as \FloatBarrier
\usepackage[section]{placeins}
% Place floats *above* footnotes
\usepackage[bottom, perpage, symbol]{footmisc}
% Set default float placement
\makeatletter
\renewcommand{\fps@figure}{tbph}
\renewcommand{\fps@table}{tbph}
\makeatother

% Pretty colours
\usepackage{xcolor}
% \usepackage{color} % Deprecated by xcolor

% SVGs with Inkscape and PDF+LaTeX
% https://tex.stackexchange.com/questions/473994/svg-and-inkscape
\usepackage[inkscapearea=page]{svg}
% Specifies the directory where vector are stored
\svgpath{{Svgs/}}

% Graphics stuff
\usepackage{graphicx}  % invoked by svg
% Specifies the directory where pictures are stored
\graphicspath{{Figures/}}

% For sub-figures and stuff
\usepackage{caption}
\usepackage{subcaption}

% Math stuff
\usepackage{amsmath} % Interesting environments
\usepackage{amssymb} % Interesting symbols
\usepackage{commath} % Interesting macros
\usepackage{braket} % Dirac bra-ket and set notations
\usepackage{mathpazo} % Math font (palatino for Computer Modern on math)
\usepackage{mathtools} % Mathematical tools to use with amsmath
\setstretch{1.5}
% Math alphabet
\DeclareMathAlphabet{\pazocal}{OMS}{zplm}{m}{n}
\newcommand{\Sa}{\pazocal{S}}
\newcommand{\Ua}{\pazocal{U}}
\newcommand{\Ha}{\pazocal{H}}
\newcommand{\Fa}{\pazocal{F}}
\newcommand{\Ia}{\pazocal{I}}
\newcommand{\Ea}{\pazocal{E}}
\newcommand{\ja}{\pazocal{J}}
%Custom math operators
\DeclareMathOperator*{\meshgrid}{meshgrid}
% Floor and ceiling of numbers
\DeclarePairedDelimiter\ceil{\lceil}{\rceil}
\DeclarePairedDelimiter\floor{\lfloor}{\rfloor}
% Notation variables
\newcommand{\dd}{\mathrm{d}}

% Units and numbers in text
\usepackage{siunitx}
\DeclareSIUnit\baud{Bd} % Baud

% Reimplementation of and extensions to LaTeX verbatim
\usepackage{verbatim} %added for \begin{comment}

% Fancy chapter start quotes
\usepackage{epigraph, varwidth}
% Overload epigraph command
\renewcommand{\epigraphsize}{\small}
\setlength{\epigraphwidth}{0.90\textwidth}
\renewcommand{\textflush}{flushright}
\renewcommand{\sourceflush}{flushright}
% A useful addition
\newcommand{\epitextfont}{\itshape}
\newcommand{\episourcefont}{\scshape}
\makeatletter
\newsavebox{\epi@textbox}
\newsavebox{\epi@sourcebox}
\newlength\epi@finalwidth
\renewcommand{\epigraph}[2]{%
  \vspace{\beforeepigraphskip}
  {\epigraphsize\begin{\epigraphflush}
   \epi@finalwidth=\z@
   \sbox\epi@textbox{%
     \varwidth{\epigraphwidth}
     \begin{\textflush}\epitextfont#1\end{\textflush}
     \endvarwidth
   }%
   \epi@finalwidth=\wd\epi@textbox
   \sbox\epi@sourcebox{%
     \varwidth{\epigraphwidth}
     \begin{\sourceflush}\episourcefont#2\end{\sourceflush}%
     \endvarwidth
   }%
   \ifdim\wd\epi@sourcebox>\epi@finalwidth
     \epi@finalwidth=\wd\epi@sourcebox
   \fi
   \leavevmode\vbox{
     \hb@xt@\epi@finalwidth{\hfil\box\epi@textbox}
     \vskip1.75ex
     \hrule height \epigraphrule
     \vskip.75ex
     \hb@xt@\epi@finalwidth{\hfil\box\epi@sourcebox}
   }%
   \end{\epigraphflush}
   \vspace{\afterepigraphskip}}}
\makeatother
%End of overload command


% Use more than one optional parameter in a new commands
\usepackage{xargs}

% Code listings
\usepackage{listings}
% Colors for the listing
\definecolor{dkgreen}{rgb}{0,0.6,0}
\definecolor{gray}{rgb}{0.5,0.5,0.5}
\definecolor{mauve}{rgb}{0.58,0,0.82}
\definecolor{codegreen}{rgb}{0,0.6,0}
\definecolor{codegray}{rgb}{0.5,0.5,0.5}
\definecolor{codepurple}{rgb}{0.58,0,0.82}
\definecolor{backcolour}{rgb}{0.95,0.95,0.92}
\definecolor{orange}{RGB}{255,127,0}
% Python style for code blocks
\lstdefinestyle{Python}{
        language=Python,
         numberstyle=\small,
         stepnumber=2,
         numbersep=10pt,
         basicstyle={\small\ttfamily},
         keywordstyle    = \color{blue},
         commentstyle    = \color{red}\ttfamily,
         stringstyle=\color{orange},
         tabsize=2,
         columns=fullflexible,
         backgroundcolor=\color{backcolour},
         frame=none,
         numbers=left,
         aboveskip=5mm,
         belowskip=5mm,
         breaklines=true
}

% Algorithmicx provides a flexible, yet easy to use, way for inserting good
% looking pseudocode or source code in your papers.
\usepackage{algorithmicx}

% Hyperref and Backref
% backref makes the bibliography say where the entry was cited.
% For the print version of the thesis you might wanna set all colors to back
\usepackage{hyperref}
\usepackage[hyperpageref]{backref}
\hypersetup{colorlinks, citecolor=blue, urlcolor=blue,
        linkcolor=blue, breaklinks=true, hypertexnames=true}
\renewcommand*{\backref}[1]{}
\renewcommand*{\backrefalt}[4]{%
    \ifcase #1%
          \or [Cited on page~#2.]%
          \else [Cited on pages~#2.]%
    \fi%
    }
% Interesting URL breakings
\usepackage{url}
\def\UrlBreaks{\do\/\do-\do\&\do.\do:}

% Variants of \fbox and other games with boxes
\usepackage{fancybox}

% LaTeX default text is fully-justified, but often left-justified text may be a
% more suitable format. This left-alignment can be easily accomplished by
% importing the ragged2e package.
\usepackage{ragged2e}

% Create tabular cells spanning multiple rows
\usepackage{multirow}

% Changes bullet points marker
\renewcommand{\labelitemi}{\(\bullet\)}

% Notes on the documents
% https://tex.stackexchange.com/questions/9796/how-to-add-todo-notes
% https://tex.stackexchange.com/questions/316220/todo-commentsnot-include-and-left-align
% Examples:
% \unsure{Is this correct?}, \change{Change this!},
% \info{This can help me in chapter seven!}
% \improvement{This really needs to be improved!\\ What was I thinking?!}
% \thiswillnotshow{This is hidden since option `disable' is chosen!}
% WARNING: It eliminates whitespaces in front of it.
% You can add trailing {} to avoid.

\usepackage[colorinlistoftodos,
    prependcaption,
    textsize=tiny,
    textwidth=2cm]
        {todonotes}
% You can add:
% \setlength{\marginparwidth}{3cm}\reversemarginpar
% before \todo on each command for a different effect
\newcommandx{\unsure}[2][1=]{
    % \setlength{\marginparwidth}{3cm}\reversemarginpar
    \todo[linecolor=red,backgroundcolor=red!25,bordercolor=red,#1]{#2}
    }
\newcommandx{\change}[2][1=]{
    % \setlength{\marginparwidth}{3cm}\reversemarginpar
    \todo[linecolor=blue,backgroundcolor=blue!25,bordercolor=blue,#1]{#2}
    }
\newcommandx{\info}[2][1=]{
    % \setlength{\marginparwidth}{3cm}\reversemarginpar
    \todo[linecolor=green,backgroundcolor=green!25,bordercolor=green,#1]{#2}
    }
\newcommandx{\improvement}[2][1=]{
    % \setlength{\marginparwidth}{3cm}\reversemarginpar
    \todo[linecolor=yellow,backgroundcolor=yellow!25,bordercolor=yellow,#1]{#2}
    }
\newcommandx{\thiswillnotshow}[2][1=]{\todo[disable,#1]{#2}}
